\pdfminorversion=4

\documentclass[10pt,aspectratio=169]{beamer}

\usetheme{CEACarre}

\setbeamercolor{background canvas}{bg=black}
\setbeamercolor{normal text}{fg=white}
\setbeamercolor{structure}{fg=white}
\setbeamercolor{title}{fg=white}
\setbeamercolor{section in toc}{fg=white}
\setbeamercolor{subsection in toc}{fg=white}
\setbeamercolor{frametitle}{fg=white}
\setbeamercolor{block title}{fg=white, bg=gray!30}
\setbeamercolor{block body}{fg=white, bg=gray!10}
\setbeamercolor{itemize item}{fg=white}
\setbeamercolor{itemize subitem}{fg=white}
\setbeamercolor{alerted text}{fg=yellow}

\usepackage[T1]{fontenc}
\usepackage[francais]{babel}
\usepackage{xcolor}
\usepackage{amsmath,amssymb,mathrsfs}
\usepackage{animate}
\usepackage[options, squaren]{SIunits}
\usepackage{adjustbox}
\usepackage{multimedia}
\usepackage{caption}
\captionsetup{labelformat=empty}
\usepackage{natbib}
\bibliographystyle{abbrvnat}
\setcitestyle{authoryear,numbers,square}
\usefonttheme{structuresmallcapsserif}
\usefonttheme[onlymath]{serif}

\usepackage{hyperref}

\renewcommand{\thesection}{\arabic{section}.}
\renewcommand{\thesubsection}{\arabic{subsection}\quad}

\newcommand{\thetitle}{\large \thesection\thesubsection \subsecname}

\usepackage{media9}

\usepackage{listings}
%\usepackage[svgnames,table]{xcolor}

% Amélioration des tableaux sur fond noir
\setbeamercolor{table}{fg=white}
\setbeamercolor{table header}{fg=white, bg=blue!40!black}
\setbeamercolor{table row}{bg=black}
\setbeamercolor{table odd row}{bg=black!80!blue}
\setbeamercolor{table even row}{bg=black!90!blue}

% ========================================
% UTILISATION DE TOUTES LES COMMANDES CEA
% ========================================

\title{Template CEA}
\subtitle{Illustration des commandes disponibles}
\date[21/05/25]{La date}
\author[John Hamon]{John \textsc{Hamon} \footnotesize (Affiliation)}

% Commande \seminar avec version courte et complète
\seminar[Démo Template]{Template CEA Beamer}

% Commandes spécifiques CEA pour la page finale
\CEAcenter{Centre}
\addr{L'adresse\\[0.6ex] France}
\mail{john.hamon@hotmail.fr}
\tel{+33 1 69 26 50 69}

\def\path{./fig}

\begin{document}

% ========================================
% PAGE DE TITRE
% ========================================
\begin{frame}[plain]
\begin{minipage}[c]{.55\linewidth}
\titlepage
\end{minipage}\hfill
\begin{minipage}[b]{.4\linewidth}
%\vspace{2cm}
%\centering \includegraphics[width=0.9\textwidth]{\path/logo_exemple}
\end{minipage}
\end{frame}

% ========================================
% LISTES À PUCES AUTOMATIQUES (4 niveaux)
% ========================================
\begin{frame}{Listes à puces automatiques}
\textbf{Les 4 niveaux d'imbrication utilisent automatiquement les marqueurs CEA :}
\begin{itemize}
    \item Premier niveau - Carré rouge CEA (\textbackslash CEAi)
    \begin{itemize}
        \item Deuxième niveau - Carré gris foncé (\textbackslash CEAii)
        \begin{itemize}
            \item Troisième niveau - Carré bleu clair (\textbackslash CEAiii)
            \begin{itemize}
                \item Quatrième niveau - Carré bleu foncé (\textbackslash CEAiv)
            \end{itemize}
        \end{itemize}
    \end{itemize}
    \item Retour au premier niveau
\end{itemize}
\end{frame}

% ========================================
% MARQUEURS PERSONNALISÉS
% ========================================
\begin{frame}{Marqueurs personnalisés pour les listes}
\textbf{Vous pouvez utiliser manuellement les marqueurs :}
\vspace{0.5em}

\CEAi{} Marqueur rouge CEA (niveau 1) - \texttt{\textbackslash CEAi}

\CEAii{} Marqueur gris foncé (niveau 2) - \texttt{\textbackslash CEAii}

\CEAiii{} Marqueur bleu clair (niveau 3) - \texttt{\textbackslash CEAiii}

\CEAiv{} Marqueur bleu foncé (niveau 4) - \texttt{\textbackslash CEAiv}

\vspace{1em}
\textbf{Marqueurs de statut :}

\done{} Tâche terminée (vert) - \texttt{\textbackslash done}

\wip{} Travail en cours (orange) - \texttt{\textbackslash wip}
\end{frame}

% ========================================
% LISTES NUMÉROTÉES
% ========================================
\begin{frame}{Listes numérotées}
\textbf{Les listes numérotées utilisent le style CEA (rouge et gras) :}
\begin{enumerate}
    \item Premier élément numéroté
    \item Deuxième élément numéroté
    \item Troisième élément numéroté
        \begin{enumerate}
            \item Sous-élément 1
            \item Sous-élément 2
        \end{enumerate}
    \item Quatrième élément numéroté
\end{enumerate}
\end{frame}

% ========================================
% BLOCS
% ========================================
\begin{frame}{Blocs colorés}

\vspace{0.5em}

\begin{alertblock}{Bloc d'alerte}
Ceci est un bloc d'alerte pour mettre en évidence des informations importantes.
\end{alertblock}

\vspace{0.5em}

\begin{exampleblock}{Bloc d'exemple}
Ceci est un bloc d'exemple pour illustrer des concepts.
\end{exampleblock}
\end{frame}

% ========================================
% COULEURS CEA DISPONIBLES
% ========================================
\begin{frame}{Palette de couleurs CEA}
\textbf{Couleurs principales :}

\textcolor{CEAred}{CEAred - Rouge CEA} $\bullet$
\textcolor{CEAdarkblue}{CEAdarkblue - Bleu foncé} $\bullet$
\textcolor{CEAlightblue}{CEAlightblue - Bleu clair}

\vspace{0.5em}
\textbf{Nuances de gris :}

\textcolor{CEAdarkgray}{CEAdarkgray - Gris foncé} $\bullet$
\textcolor{CEAgray}{CEAgray - Gris} $\bullet$
\textcolor{CEAlightgray}{CEAlightgray - Gris clair (sur fond blanc)}

\vspace{0.5em}
\textbf{Couleurs complémentaires :}

\textcolor{CEAyellow}{CEAyellow - Jaune} $\bullet$
\textcolor{CEAmacaron}{CEAmacaron - Macaron} $\bullet$
\textcolor{CEAarchipel}{CEAarchipel - Archipel}

\textcolor{CEAopera}{CEAopera - Opera} $\bullet$
\textcolor{CEAglycine}{CEAglycine - Glycine} $\bullet$
\textcolor{Green}{Green - Vert} $\bullet$
\textcolor{Orange}{Orange - Orange}
\end{frame}

% ========================================
% TEXTE ACCENTUÉ
% ========================================
\begin{frame}{Texte accentué}
\textbf{Mise en évidence du texte :}

Texte normal avec \alert{texte en alerte} qui apparaît en jaune CEA.

\vspace{1em}

Vous pouvez aussi utiliser \textcolor{CEAyellow}{des couleurs personnalisées} dans votre texte.

\vspace{1em}

\textbf{Texte en gras} et \textit{texte en italique} fonctionnent normalement.
\end{frame}

% ========================================
% UTILISATION DES COMMANDES INSÉRÉES
% ========================================
\begin{frame}{Informations récupérées automatiquement}
\textbf{Le template utilise automatiquement :}
\begin{itemize}
    \item Auteur court dans le pied de page : \insertshortauthor
    \item Séminaire court dans le pied de page : \insertshortseminar
    \item Date courte dans le pied de page : \insertshortdate
    \item Numéro de frame actuel : \insertframenumber
\end{itemize}

\vspace{1em}
Ces informations sont définies avec :
\begin{itemize}
    \item \texttt{\textbackslash author[court]\{complet\}}
    \item \texttt{\textbackslash seminar[court]\{complet\}}
    \item \texttt{\textbackslash date[court]\{complet\}}
\end{itemize}
\end{frame}

% ========================================
% EXEMPLE AVEC COLONNES
% ========================================
\begin{frame}{Mise en page avec colonnes}
\begin{columns}[T]
\begin{column}{0.48\textwidth}
\textbf{Colonne gauche}
\begin{itemize}
    \item Point 1
    \item Point 2
    \item Point 3
\end{itemize}
\end{column}

\begin{column}{0.48\textwidth}
\textbf{Colonne droite}
\begin{enumerate}
    \item Élément 1
    \item Élément 2
    \item Élément 3
\end{enumerate}
\end{column}
\end{columns}
\end{frame}

% ========================================
% PAGE FINALE PERSONNALISÉE
% ========================================
\begin{frame}[plain]
\usebeamertemplate{last page}
\end{frame}

\end{document}